\cleardoublepage
\phantomsection
\addcontentsline{toc}{chapter}{Error Handling and Go}
\chapter*{Error Handling and Go}

If you have written any Go code you have probably encountered the
built-in \texttt{error} type. Go code uses \texttt{error} values to
indicate an abnormal state. For example, the \texttt{os.Open} function
returns a non-nil \texttt{error} value when it fails to open a file.

\begin{Verbatim}[frame=single]
func Open(name string) (file *File, err error)
\end{Verbatim}

The following code uses \texttt{os.Open} to open a file. If an error
occurs it calls \texttt{log.Fatal} to print the error message and stop.

\begin{Verbatim}[frame=single]
        f, err := os.Open("filename.ext")
        if err != nil {
                log.Fatal(err)
        }
        // do something with the open *File f
\end{Verbatim}

You can get a lot done in Go knowing just this about the \texttt{error}
type, but in this article we'll take a closer look at \texttt{error} and
discuss some good practices for error handling in Go.

\textbf{The error type}

The \texttt{error} type is an interface type. An \texttt{error} variable
represents any value that can describe itself as a string. Here is the
interface's declaration:

\begin{Verbatim}[frame=single]
type error interface {
        Error() string
}
\end{Verbatim}

The \texttt{error} type, as with all built in types, is
\href{http://golang.org/doc/go\_spec.html\#Predeclared\_identifiers}{predeclared} in the
\href{http://golang.org/doc/go\_spec.html\#Blocks}{universe block}.

The most commonly-used \texttt{error} implementation is the
\href{http://golang.org/pkg/errors/}{errors} package's unexported \texttt{errorString}
type.

\begin{Verbatim}[frame=single]
// errorString is a trivial implementation of error.
type errorString struct {
        s string
}

func (e *errorString) Error() string {
        return e.s
}
\end{Verbatim}

You can construct one of these values with the \texttt{errors.New}
function. It takes a string that it converts to an
\texttt{errors.errorString} and returns as an \texttt{error} value.

\begin{Verbatim}[frame=single]
// New returns an error that formats as the given text.
func New(text string) error {
        return &errorString{text}
}
\end{Verbatim}

Here's how you might use \texttt{errors.New}:

\begin{Verbatim}[frame=single]
func Sqrt(f float64) (float64, error) {
        if f < 0 {
                return 0, errors.New("math: square root of negative number")
        }
        // implementation
}
\end{Verbatim}

A caller passing a negative argument to \texttt{Sqrt} receives a non-nil
\texttt{error} value (whose concrete representation is an
\texttt{errors.errorString} value). The caller can access the error
string (``math: square root of\ldots{}'') by calling the
\texttt{error}'s \texttt{Error} method, or by just printing it:

\begin{Verbatim}[frame=single]
        f, err := Sqrt(-1)
        if err != nil {
                fmt.Println(err)
        }
\end{Verbatim}

The \href{http://golang.org/pkg/fmt/}{fmt} package formats an
\texttt{error} value by calling its \texttt{Error() string} method.

It is the error implementation's responsibility to summarize the
context. The error returned by \texttt{os.Open} formats as ``open
/etc/passwd: permission denied,'' not just ``permission denied.''
The error returned by our \texttt{Sqrt} is missing information about
the invalid argument.

To add that information, a useful function is the \texttt{fmt}
package's \texttt{Errorf}. It formats a string according to
\texttt{Printf}'s rules and returns it as an \texttt{error} created
by \texttt{errors.New}.

\begin{Verbatim}[frame=single]
        if f < 0 {
                return 0, fmt.Errorf("math: square root of negative number %g", f)
        }
\end{Verbatim}

In many cases \texttt{fmt.Errorf} is good enough, but since
\texttt{error} is an interface, you can use arbitrary data structures as
error values, to allow callers to inspect the details of the error.

For instance, our hypothetical callers might want to recover the invalid
argument passed to \texttt{Sqrt}. We can enable that by defining a new
error implementation instead of using \texttt{errors.errorString}:

\begin{Verbatim}[frame=single]
type NegativeSqrtError float64

func (f NegativeSqrtError) Error() string {
        return fmt.Sprintf("math: square root of negative number %g", float64(f))
}
\end{Verbatim}

A sophisticated caller can then use a
\href{http://golang.org/doc/go\_spec.html\#Type\_assertions}{type
assertion} to check for a \texttt{NegativeSqrtError} and handle it
specially, while callers that just pass the error to \texttt{fmt.Println}
or \texttt{log.Fatal} will see no change in behavior.

As another example, the \href{http://golang.org/pkg/encoding/json/}{json}
package specifies a \texttt{SyntaxError} type that the \texttt{json.Decode}
function returns when it encounters a syntax error parsing a JSON
blob.

\begin{Verbatim}[frame=single]
type SyntaxError struct {
        msg    string // description of error
        Offset int64  // error occurred after reading Offset bytes
}

func (e *SyntaxError) Error() string { return e.msg }
\end{Verbatim}

The \texttt{Offset} field isn't even shown in the default formatting of
the error, but callers can use it to add file and line information to
their error messages:

\begin{Verbatim}[frame=single]
        if err := dec.Decode(&val); err != nil {
                if serr, ok := err.(*json.SyntaxError); ok {
                        line, col := findLine(f, serr.Offset)
                        return fmt.Errorf("%s:%d:%d: %v", f.Name(), line, col, err)
                }
                return err
        }
\end{Verbatim}

(This is a slightly simplified version of some
\href{http://camlistore.org/code/?p=camlistore.git;a=blob;f=lib/go/camli/jsonconfig/eval.go\#l68}{actual
code} from the \href{http://camlistore.org}{Camlistore} project.)

The \texttt{error} interface requires only a \texttt{Error} method;
specific error implementations might have additional methods. For
instance, the \href{http://golang.org/pkg/net/}{net} package returns errors of type
\texttt{error}, following the usual convention, but some of the error
implementations have additional methods defined by the
\texttt{net.Error} interface:

\begin{Verbatim}[frame=single]
package net

type Error interface {
    error
    Timeout() bool   // Is the error a timeout?
    Temporary() bool // Is the error temporary?
}
\end{Verbatim}

Client code can test for a \texttt{net.Error} with a type assertion and
then distinguish transient network errors from permanent ones. For
instance, a web crawler might sleep and retry when it encounters a
temporary error and give up otherwise.

\begin{Verbatim}[frame=single]
        if nerr, ok := err.(net.Error); ok && nerr.Temporary() {
                time.Sleep(1e9)
                continue
        }
        if err != nil {
                log.Fatal(err)
        }
\end{Verbatim}

\section*{Simplifying repetitive error handling}

In Go, error handling is important. The language's design and
conventions encourage you to explicitly check for errors where they
occur (as distinct from the convention in other languages of throwing
exceptions and sometimes catching them). In some cases this makes Go
code verbose, but fortunately there are some techniques you can use to
minimize repetitive error handling.

Consider an \href{http://code.google.com/appengine/docs/go/}{App Engine}
application with an HTTP handler that retrieves a record from the
datastore and formats it with a template.

\begin{Verbatim}[frame=single]
func init() {
        http.HandleFunc("/view", viewRecord)
}

func viewRecord(w http.ResponseWriter, r *http.Request) {
        c := appengine.NewContext(r)
        key := datastore.NewKey(c, "Record", r.FormValue("id"), 0, nil)
        record := new(Record)
        if err := datastore.Get(c, key, record); err != nil {
                http.Error(w, err.Error(), 500)
                return
        }
        if err := viewTemplate.Execute(w, record); err != nil {
                http.Error(w, err.Error(), 500)
        }
}
\end{Verbatim}

This function handles errors returned by the \texttt{datastore.Get}
function and \texttt{viewTemplate}'s \texttt{Execute} method. In both
cases, it presents a simple error message to the user with the HTTP
status code 500 (``Internal Server Error''). This looks like a
manageable amount of code, but add some more HTTP handlers and you
quickly end up with many copies of identical error handling code.

To reduce the repetition we can define our own HTTP \texttt{appHandler}
type that includes an \texttt{error} return value:

\begin{Verbatim}[frame=single]
type appHandler func(http.ResponseWriter, *http.Request) error
\end{Verbatim}

Then we can change our \texttt{viewRecord} function to return errors:

\begin{Verbatim}[frame=single]
func viewRecord(w http.ResponseWriter, r *http.Request) error {
        c := appengine.NewContext(r)
        key := datastore.NewKey(c, "Record", r.FormValue("id"), 0, nil)
        record := new(Record)
        if err := datastore.Get(c, key, record); err != nil {
                return err
        }
        return viewTemplate.Execute(w, record)
}
\end{Verbatim}

This is simpler than the original version, but the
\href{http://golang.org/pkg/net/http/}{http} package doesn't understand functions that
return \texttt{error}. To fix this we can implement the
\texttt{http.Handler} interface's \texttt{ServeHTTP} method on
\texttt{appHandler}:

\begin{Verbatim}[frame=single]
func (fn appHandler) ServeHTTP(w http.ResponseWriter, r *http.Request) {
        if err := fn(w, r); err != nil {
                http.Error(w, err.Error(), 500)
        }
}
\end{Verbatim}

The \texttt{ServeHTTP} method calls the \texttt{appHandler} function and
displays the returned error (if any) to the user. Notice that the
method's receiver, \texttt{fn}, is a function. (Go can do that!) The
method invokes the function by calling the receiver in the expression
\texttt{fn(w, r)}.

Now when registering \texttt{viewRecord} with the http package we use
the \texttt{Handle} function (instead of \texttt{HandleFunc}) as
\texttt{appHandler} is an \texttt{http.Handler} (not an
\texttt{http.HandlerFunc}).

\begin{Verbatim}[frame=single]
func init() {
        http.Handle("/view", appHandler(viewRecord))
}
\end{Verbatim}

With this basic error handling infrastructure in place, we can make it
more user friendly. Rather than just displaying the error string, it
would be better to give the user a simple error message with an
appropriate HTTP status code, while logging the full error to the App
Engine developer console for debugging purposes.

To do this we create an \texttt{appError} struct containing an
\texttt{error} and some other fields:

\begin{Verbatim}[frame=single]
type appError struct {
        Error   error
        Message string
        Code    int
}
\end{Verbatim}

Next we modify the appHandler type to return \texttt{*appError} values:

\begin{Verbatim}[frame=single]
type appHandler func(http.ResponseWriter, *http.Request) *appError
\end{Verbatim}

(It's usually a mistake to pass back the concrete type of an error
rather than \texttt{error}, for reasons discussed in
\href{http://golang.org/doc/go\_faq.html\#nil\_error}{the Go FAQ}, but it's the right
thing to do here because \texttt{ServeHTTP} is the only place that sees
the value and uses its contents.)

And make \texttt{appHandler}'s \texttt{ServeHTTP} method display the
\texttt{appError}'s \texttt{Message} to the user with the correct HTTP
status \texttt{Code} and log the full \texttt{Error} to the developer
console:

\begin{Verbatim}[frame=single]
func (fn appHandler) ServeHTTP(w http.ResponseWriter, r *http.Request) {
        if e := fn(w, r); e != nil { // e is *appError, not os.Error.
                c := appengine.NewContext(r)
                c.Errorf("%v", e.Error)
                http.Error(w, e.Message, e.Code)
        }
}
\end{Verbatim}

Finally, we update \texttt{viewRecord} to the new function signature and
have it return more context when it encounters an error:

\begin{Verbatim}[frame=single]
func viewRecord(w http.ResponseWriter, r *http.Request) *appError {
    c := appengine.NewContext(r)
    key := datastore.NewKey(c, "Record", r.FormValue("id"), 0, nil)
    record := new(Record)
    if err := datastore.Get(c, key, record); err != nil {
        return &appError{err, "Record not found", 404}
    }
    if err := viewTemplate.Execute(w, record); err != nil {
        return &appError{err, "Can't display record", 500}
    }
    return nil
}
\end{Verbatim}

This version of \texttt{viewRecord} is the same length as the original,
but now each of those lines has specific meaning and we are providing a
friendlier user experience.

It doesn't end there; we can further improve the error handling in our
application. Some ideas:

\begin{itemize}
\item
  give the error handler a pretty HTML template,
\item
  make debugging easier by writing the stack trace to the HTTP response
  when the user is an administrator,
\item
  write a constructor function for \texttt{appError} that stores the
  stack trace for easier debugging,
\item
  recover from panics inside the \texttt{appHandler}, logging the error
  to the console as ``Critical,'' while telling the user ``a serious
  error has occurred.'' This is a nice touch to avoid exposing the user
  to inscrutable error messages caused by programming errors. See the
  \href{defer\_panic\_recover.html}{Defer, Panic, and Recover} article
  for more details.
\end{itemize}

\section*{Conclusion}

Proper error handling is an essential requirement of good software. By
employing the techniques described in this post you should be able to
write more reliable and succinct Go code.
