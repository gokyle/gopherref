\cleardoublepage
\phantomsection
\addcontentsline{toc}{chapter}{How to Write Go Code}
\chapter*{How to Write Go Code}

\section *{Introduction}
This document demonstrates the development of a simple Go package
and introduces the \href{http://golang.org/cmd/go/}{go} command,
the standard way to fetch, build, and install Go packages and
commands.

\section*{Code organization}
\subsection*{GOPATH and workspaces}

One of Go's design goals is to make writing software easier. To
that end, the go command doesn't use Makefiles or other configuration
files to guide program construction. Instead, it uses the source
code to find dependencies and determine build conditions. This means
your source code and build scripts are always in sync; they are one
and the same.

The one thing you must do is set a \verb|GOPATH| environment variable.
\verb|GOPATH| tells the go command (and other related tools) where to find
and install the Go packages on your system.

\verb|GOPATH| is a list of paths. It shares the syntax of your
system's \verb|PATH| environment variable. A typical \verb|GOPATH|
on a Unix system might look like this:

\begin{Verbatim}[frame=single]
GOPATH=/home/user/ext:/home/user/mygo
\end{Verbatim}

(On a Windows system use semicolons as the path separator instead of colons.)

Each path in the list (in this case \verb|/home/user/ext| or
\verb|/home/user/mygo|) specifies the location of a \textit{workspace}.
A workspace contains Go source files and their associated package
objects, and command executables. It has a prescribed structure of
three subdirectories:

\begin{itemize}
  \item \verb|src| contains Go source files,
  \item \verb|pkg| contains compiled package objects, and
  \item \verb|bin| contains executable commands.
\end{itemize}

Subdirectories of the \verb|src| directory hold independent packages,
and all source files (\verb|.go, .c, .h, and .s|) in each subdirectory
are elements of that subdirectory's package.

When building a program that imports the package "\verb|widget|"
the \verb|go| command looks for \verb|src/pkg/widget| inside the
Go root, and then—if the package source isn't found there—it searches
for \verb|src/widget| inside each workspace in order.

Multiple workspaces can offer some flexibility and convenience, but
for now we'll concern ourselves with only a single workspace.

Let's work through a simple example. First, create a \verb|$HOME/mygo|
directory and its \verb|src| subdirectory:

\begin{Verbatim}[frame=single]
$ mkdir -p $HOME/mygo/src # create a place to put source code
\end{Verbatim}

Next, set it as the \verb|GOPATH|. You should also add the \verb|bin|
subdirectory to your \verb|PATH| environment variable so that you
can run the commands therein without specifying their full path.
To do this, add the following lines to \verb|$HOME/.profile| (or
equivalent):

\begin{Verbatim}[frame=single]
export GOPATH=$HOME/mygo
export PATH=$PATH:$HOME/mygo/bin
\end{Verbatim}

\subsection*{Import paths}
The standard packages are given short import paths such as "\verb|fmt|"
and "\verb|net/http|" for convenience. For your own projects, it
is important to choose a base import path that is unlikely to collide
with future additions to the standard library or other external
libraries.

The best way to choose an import path is to use the location of
your version control repository. For instance, if your source
repository is at \verb|example.com| or \verb|code.google.com/p/example|,
you should begin your package paths with that URL, as in
"\verb|example.com/foo/bar|" or "\verb|code.google.com/p/example/foo/bar|".
Using this convention, the \verb|go| command can automatically check
out and build the source code by its import path alone.

If you don't intend to install your code in this way, you should
at least use a unique prefix like "\verb|widgets/|", as in
"\verb|widgets/foo/bar|". A good rule is to use a prefix such as
your company or project name, since it is unlikely to be used by
another group.

We'll use example/ as our base import path:

\begin{Verbatim}[frame=single]
$ mkdir -p $GOPATH/src/example
\end{Verbatim}

\subsection*{Package names}
The first statement in a Go source file should be

\begin{Verbatim}[frame=single]
package name
\end{Verbatim}

where \textit{name} is the package's default name for imports. (All
files in a package must use the same \textit{name}.)

Go's convention is that the package name is the last element of the
import path: the package imported as "\verb|crypto/rot13|" should
be named \verb|rot13|. There is no requirement that package names
be unique across all packages linked into a single binary, only
that the import paths (their full file names) be unique.

Create a new package under \verb|example| called \verb|newmath|:

\begin{Verbatim}[frame=single]
$ cd $GOPATH/src/example
$ mkdir newmath
\end{Verbatim}

Then create a file named \verb|$GOPATH/src/example/newmath/sqrt.go|
containing the following Go code:

\begin{Verbatim}[frame=single]
// Package newmath is a trivial example package.
package newmath

// Sqrt returns an approximation to the square root of x.
func Sqrt(x float64) float64 {
        // This is a terrible implementation.
        // Real code should import "math" and use math.Sqrt.
        z := 0.0
        for i := 0; i < 1000; i++ {
                z -= (z*z - x) / (2 * x)
        }
        return z
}
\end{Verbatim}

This package is imported by the path name of the directory it's in,
starting after the \verb|src| component:

\begin{Verbatim}[frame=single]
import "example/newmath"
\end{Verbatim}

See \href{http://golang.org/doc/effective_go.html#names}{Effective
Go} to learn more about Go's naming conventions.

\section*{Building and Installing}
The \verb|go| command comprises several subcommands, the most central
being \verb|install|. Running \verb|go install| \textit{importpath}
builds and installs a package and its dependencies.

To "install a package" means to write the package object or executable
command to the \verb|pkg| or \verb|bin| subdirectory of the workspace
in which the source resides.

\subsection*{Building a package}

To build and install the newmath package, type

\begin{Verbatim}[frame=single]
$ go install example/newmath
\end{Verbatim}

This command will produce no output if the package and its dependencies
are built and installed correctly.

As a convenience, the \verb|go| command will assume the current
directory if no import path is specified on the command line. This
sequence of commands has the same effect as the one above:

\begin{Verbatim}[frame=single]
$ cd $GOPATH/src/example/newmath
$ go install
\end{Verbatim}

The resulting workspace directory tree (assuming we're running Linux
on a 64-bit system) looks like this:

\begin{Verbatim}[frame=single]
pkg/
    linux_amd64/
        example/
            newmath.a  # package object
src/
    example/
        newmath/
            sqrt.go    # package source
\end{Verbatim}

\subsection*{Building a command}
The \verb|go| command treats code belonging to package main as an
executable command and installs the package binary to the \verb|GOPATH|'s
\verb|bin| subdirectory.

Add a command named \verb|hello| to the source tree. First create
the \verb|example/hello| directory:

\begin{Verbatim}[frame=single]
$ cd $GOPATH/src/example
$ mkdir hello
\end{Verbatim}

Then create the file \verb|$GOPATH/src/example/hello/hello.go|
containing the following Go code.

\begin{Verbatim}[frame=single]
// Hello is a trivial example of a main package.
package main

import (
        "example/newmath"
        "fmt"
)

func main() {
        fmt.Printf("Hello, world.  Sqrt(2) = %v\n", newmath.Sqrt(2))
}
\end{Verbatim}

Next, run \verb|go install|, which builds and installs the binary
to \verb|$GOPATH/bin| (or \verb|$GOBIN|, if set; to simplify
presentation, this document assumes \verb|GOBIN| is unset):

\begin{Verbatim}[frame=single]
$ go install example/hello
\end{Verbatim}

To run the program, invoke it by name as you would any other command:

\begin{Verbatim}[frame=single]
$ $GOPATH/bin/hello
Hello, world.  Sqrt(2) = 1.414213562373095
\end{Verbatim}

If you added \verb|$HOME/mygo/bin| to your \verb|PATH|, you may
omit the path to the executable:

\begin{Verbatim}[frame=single]
$ hello
Hello, world.  Sqrt(2) = 1.414213562373095
\end{Verbatim}

The workspace directory tree now looks like this:

\begin{Verbatim}[frame=single]
bin/
    hello              # command executable
pkg/
    linux_amd64/ 
        example/
            newmath.a  # package object
src/
    example/
        hello/
            hello.go   # command source
        newmath/
            sqrt.go    # package source
\end{Verbatim}

The \verb|go| command also provides a \verb|build| command, which
is like \verb|install| except it builds all objects in a temporary
directory and does not install them under \verb|pkg| or \verb|bin|.
When building a command an executable named after the last element
of the import path is written to the current directory. When building
a package, \verb|go build| serves merely to test that the package
and its dependencies can be built. (The resulting package object
is thrown away.)

\section*{Testing}
Go has a lightweight test framework composed of the \verb|go test|
command and the \verb|testing| package.

You write a test by creating a file with a name ending in \verb|_test.go|
that contains functions named \verb|TestXXX| with signature 
\verb|func (t *testing.T)|. The test framework runs each such function;
if the function calls a failure function such as \verb|t.Error| or
\verb|t.Fail|, the test is considered to have failed.

Add a test to the \verb|newmath| package by creating the file
\verb|$GOPATH/src/example/newmath/sqrt_test.go| containing the
following Go code.

\begin{Verbatim}[frame=single]
package newmath

import "testing"

func TestSqrt(t *testing.T) {
        const in, out = 4, 2
        if x := Sqrt(in); x != out {
                t.Errorf("Sqrt(%v) = %v, want %v", in, x, out)
        }
}
\end{Verbatim}

Now run the test with \verb|go test|:

\begin{Verbatim}[frame=single]
$ go test example/newmath
ok      example/newmath 0.165s
\end{Verbatim}

Run go help test and see the \href{http://golang.org/pkg/testing/}{testing
package documentation} for more detail.

\section*{Remote packages}
An import path can describe how to obtain the package source code
using a revision control system such as Git or Mercurial. The go
command uses this property to automatically fetch packages from
remote repositories. For instance, the examples described in this
document are also kept in a Mercurial repository hosted at Google
Code, code.google.com/p/go.example. If you include the repository
URL in the package's import path, \verb|go get| will fetch, build,
and install it automatically:

\begin{Verbatim}[frame=single]
$ go get code.google.com/p/go.example/hello
$ $GOPATH/bin/hello
Hello, world.  Sqrt(2) = 1.414213562373095
\end{Verbatim}

If the specified package is not present in a workspace, go get will
place it inside the first workspace specified by \verb|GOPATH|. (If
the package does already exist, \verb|go get| skips the remote fetch
and behaves the same as \verb|go install|.)

After issuing the above \verb|go geti| command, the workspace
directory tree should now now look like this:

\begin{Verbatim}[frame=single]
bin/
    hello                 # command executable
pkg/
    linux_amd64/ 
        code.google.com/p/go.example/
            newmath.a     # package object
        example/
            newmath.a     # package object
src/
     code.google.com/p/go.example/
         hello/
             hello.go      # command source
         newmath/
             sqrt.go       # package source
             sqrt_test.go  # test source
     example/
         hello/
             hello.go      # command source
         newmath/
             sqrt.go       # package source
             sqrt_test.go  # test source
\end{Verbatim}

The \verb|hello| command hosted at Google Code depends on the
\verb|newmath| package within the same repository. The imports in
\verb|hello.go| file use the same import path convention, so the
\verb|go get| command is able to locate and install the dependent
package, too.

\begin{Verbatim}[frame=single]
import "code.google.com/p/go.example/newmath"
\end{Verbatim}

This convention is the easiest way to make your Go packages available
for others to use. The \href{http://godashboard.appspot.com/}{Go
Project Dashboard} is a list of external Go projects including
programs and libraries.

For more information on using remote repositories with the \verb|go|
command, see \verb|go help remote|.

\section*{Further reading}

See \href{http://golang.org/doc/effective_go.html}{Effective Go}
for tips on writing clear, idiomatic Go code.

\noindent Take \href{http://tour.golang.org/}{A Tour of Go} to learn the
language proper.

\noindent Visit the \href{http://golang.org/doc/#articles}{documentation page}
for a set of in-depth articles about the Go language and its libraries
and tools.
