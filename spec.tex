\documentclass{article}

\title{The Go Language Specification}
\author{Go 1.0.3}
\date{2012 September 23}

\begin{document}
\maketitle

\section*{Introduction}
This is a reference manual for the Go programming language. For
more information and other documents, see \url{http://golang.org}.

Go is a general-purpose language designed with systems programming
in mind. It is strongly typed and garbage-collected and has explicit
support for concurrent programming. Programs are constructed from
packages, whose properties allow efficient management of dependencies.
The existing implementations use a traditional compile/link model
to generate executable binaries.

The grammar is compact and regular, allowing for easy analysis by
automatic tools such as integrated development environments.

\section*{Notation}

The syntax is specified using Extended Backus-Naur Form (EBNF):

\begin{Verbatim}[frame=single]
Production  = production_name "=" [ Expression ] "." .
Expression  = Alternative { "|" Alternative } .
Alternative = Term { Term } .
Term        = production_name | token [ "…" token ]
                              | Group | Option | Repetition .
Group       = "(" Expression ")" .
Option      = "[" Expression "]" .
Repetition  = "{" Expression "}" .
\end{Verbatim}

Productions are expressions constructed from terms and the following operators, in increasing precedence:

\begin{Verbatim}[frame=single]
|   alternation
()  grouping
[]  option (0 or 1 times)
{}  repetition (0 to n times)
\end{Verbatim}

Lower-case production names are used to identify lexical tokens.
Non-terminals are in CamelCase. Lexical tokens are enclosed in
double quotes "" or back quotes ``.

The form a $\ldots$ b represents the set of characters from a through b
as alternatives. The horizontal ellipsis $\ldots$ is also used elsewhere
in the spec to informally denote various enumerations or code
snippets that are not further specified. The character $\ldots$ (as opposed
to the three characters ...) is not a token of the Go language.


\section*{Source code representation}
Source code is Unicode text encoded in
\href{UTF-8}{http://en.wikipedia.org/wiki/UTF-8}. The text is
not canonicalized, so a single accented code point is distinct from
the same character constructed from combining an accent and a letter;
those are treated as two code points. For simplicity, this document
will use the unqualified term character to refer to a Unicode code
point in the source text.

Each code point is distinct; for instance, upper and lower case
letters are different characters.

Implementation restriction: For compatibility with other tools, a
compiler may disallow the NUL character (U+0000) in the source text.

\subsection*{Characters}
The following terms are used to denote specific Unicode character
classes:

\begin{Verbatim}[frame=single]
newline        = /* the Unicode code point U+000A */ .
unicode_char   = /* an arbitrary Unicode code point except newline */ .
unicode_letter = /* a Unicode code point classified as "Letter" */ .
unicode_digit  = /* a Unicode code point classified as "Decimal Digit" */ .
\end{Verbatim}

In \href{http://www.unicode.org/versions/Unicode6.0.0/}{The Unicode Standard 6.0},
Section 4.5 "General Category" defines a set of character categories.
Go treats those characters in category Lu, Ll, Lt, Lm, or Lo as
Unicode letters, and those in category Nd as Unicode digits.

\subsection*{Letters and Digits}
The underscore character \_ (U+005F) is considered a letter.

\begin{Verbatim}[frame=single]
letter        = unicode_letter | "_" .
decimal_digit = "0" … "9" .
octal_digit   = "0" … "7" .
hex_digit     = "0" … "9" | "A" … "F" | "a" … "f" .
\end{Verbatim}

\section*{Lexical elements}

\subsection*{Comments}
There are two forms of comments:

\begin{enumerate}
  \item \textit{Line comments} start with the character sequence
  // and stop at the end of the line. A line comment acts like a
  newline.
  
  \item \textit{General comments} start with the character
  sequence /* and continue through the character sequence */. A
  general comment containing one or more newlines acts like a
  newline, otherwise it acts like a space.
\end{enumerate}

Comments do not nest.

\subsection*{Tokens}
Tokens form the vocabulary of the Go language. There are four
classes: identifiers, keywords, operators and delimiters, and
literals. White space, formed from spaces (U+0020), horizontal tabs
(U+0009), carriage returns (U+000D), and newlines (U+000A), is
ignored except as it separates tokens that would otherwise combine
into a single token. Also, a newline or end of file may trigger the
insertion of a semicolon. While breaking the input into tokens, the
next token is the longest sequence of characters that form a valid
token.

\subsection*{Semicolons}
The formal grammar uses semicolons ";" as terminators in a number
of productions. Go programs may omit most of these semicolons using
the following two rules:

\begin{enumerate}

  \item When the input is broken into tokens, a semicolon is
  automatically inserted into the token stream at the end of a
  non-blank line if the line's final token is

  \begin{itemize}

    \item an identifier

    \item an integer, floating-point, imaginary, rune, or string
    literal

    \item one of the keywords break, continue, fallthrough, or
    return

    \item one of the operators and delimiters ++, --, ), ], or \}
  \end{itemize}

  \item To allow complex statements to occupy a single line, a
  semicolon may be omitted before a closing ")" or "\}".

\end{enumerate}

To reflect idiomatic use, code examples in this document elide
semicolons using these rules.

\subsection*{Identifiers}
Identifiers name program entities such as variables and types. An
identifier is a sequence of one or more letters and digits. The
first character in an identifier must be a letter.

\begin{Verbatim}[frame=single]
identifier = letter { letter | unicode_digit } .
\end{Verbatim}

\begin{Verbatim}[frame=single]
a
_x9
ThisVariableIsExported
αβ
\end{Verbatim}

Some identifiers are predeclared.


\end{document}
